\documentclass[14pt,a4paper]{scrartcl}
 
\usepackage[utf8]{inputenc}
\usepackage[T1]{fontenc}
\usepackage{lmodern}
\usepackage[ngerman]{babel}
\usepackage{amsmath}

\title{Manual}
\author{Frederik und Henrik}
\date{\today}
\begin{document}
 
\maketitle
\tableofcontents
\newpage
\section{Inhalt}
ProjektarbeitDiffusion C++ Projekt:

\subsection{Sourcedateien}
Teilchen.h;
Teilchen.cpp;
Kasten.h;
Kasten.cpp;
Teilchengernerator.h;
Teilchengernerator.cpp;
Ausgleich.h;
Ausgleich.cpp;
Eingaben.h;
Eingaben.cpp;
Grafikoutput.h;
Grafikoutput.cpp;
ProjektarbeitDiffusion.cpp;

\subsection{Diffusion}
Unser Programm soll den Diffusionsvorgang veranschaulichen.
Es werden wahlweise in einem Kasten, der durch eine Wand mit einem Spalt in zwei Hälften unterteilt ist, zufällig oder manuell Teilchen erzeugt.

\subsection{Programm starten}
Installieren sie das Programm mit CMake und führen sie die Executable aus.
Folgen sie nun den Instruktionen des Programms.

\section{Features}
\textbf{Mögliche Einstellungen:}

\subsection{Einstellungen Teilchen}

\begin{enumerate}

\item \textbf{Teilchenanzahl}
\newline
Die Anzahl der Teilchen, die in die Simulation Einfließen sollen, muss zu Beginn festgelegt werden.

\item \textbf{Teilchen automatisch erstellen}

\begin{itemize}

\item Position
\newline
Die x und y Koordinaten der automatisch erzeugten Teilchen sind zufällig im rechten Teil des Kastens verteilt, wobei ein Mindestabstand des Mittelpunktes der Teilchen von einem zehntel der Spaltbreite festgelegt ist.

\item Geschwindigkeit
\newline
Die Geschwindigkeiten der automatisch erzeugten Teilchen sind in beiden Richtungen zufällig verteilt und beschränkt durch:
\begin{align*}
(\frac{1}{20}-\frac{1}{40})*\text{Kastengröße}
\end{align*}
Damit ist die Kastenbreite(v$_x$) bzw. Kastenhöhe(v$_y$)
gemeint. Die Beschränkung kann im Quellcode "Teilchengenerator.cpp" geändert werden, wobei zu große Geschwindigkeiten bei zu großem dt zu Fehlern führen.
\item Radius auto
\newline
Falls der Radius auf automatisch generieren gesetzt wird, werden Zufallszahlen zwischen 0.01 und ca. einem Zehntel der Spaltbreite generiert.
\item Radius fest
\newline
Der Radius kann beliebig gewählt werden und wird für alle folgenden Iterationen verwendet.
\item Masse auto
\newline
Falls die Masse auf automatisch generieren gesetzt wird,
werden Zufallszahlen zwischen 0.01 und ca. einem Zehntel der Spaltbreite generiert. (Die Beschränkungen für Radius und Masse können im Quellcode "Teilchengenerator.cpp" konfiguriert werden.  Änderungen können zu Fehlern bei der Ausführung des Programmes führen!)
\item Masse fest
\newline
Die Masse kann beliebig gewählt werden und wird für alle folgenden Iterationen verwendet. 

\end{itemize}
\end{enumerate}

\subsection{Einstellungen "dt" }
"dt" bzw. den "Zeitschritt", der für die Berechnungen benutzt wird kann man beliebig wählen, wobei zu große dt zu Fehlern bei der Berechnung der Kollisionen führen können.
Zu kleine dts führen zu sehr großem Rechenaufwand und im Extremfall zu Rundungsfehlern.

\subsection{Anzahl der Simulationen}
Die Anzahl der Durchläufe oder Simulationen, die mit den gleichen Bedingungen durchgeführt werden sollen, kann kurz nach Start des Programmes festgelegt werden.

\subsection{Einstellungen Kasten}
\begin{enumerate}

\item \textbf{Default Kasten}
\newline
Bevor die Simulation startet, kann gewählt werden, ob man selber einen Kasten "bauen" möchte oder ob mit dem Standartkasten( Breite 100, Höhe 100, Spaltbreite 20 ) simuliert werden soll.
\footnote{"Die rechte Seite des Kastens ist nur durch den Spalt mit der linken Seite verbunden"}
\item \textbf{Kasten von Hand erstellen}
\newline
Wenn man den Kasten selber erstellen möchte, kann man Höhe, Breite und Spaltbreite, nachdem man "selber Hand anlegen" gewählt hat, einstellen.

\end{enumerate}

\subsection{Simulationen Plotten}
Falls die Simulation geplottet werden soll, (es wird mindesten Gnuplot Version 4.6 benötigt)
kann man das vor dem Start des Rechenprozesses auswählen.
Damit ist gemeint, dass eine Gif erstellt wird, ein Histogramm und eine Ausgleichskurve, womit die berechneten Daten anschaulich dargestellt werden können.
Die Gif Datei wird jedes Interation überschrieben.
Falls man dies vermeiden möchte sollte man nach der jeweiligen Iteration die Gif Datei in ein anderes Verzeichnis kopieren.
Da die Rechenzeit für das erstellen des Gifs je nach Wahl der Anfangsbedingungen relativ hoch ist, empfiehlet es sich nur einen Plot zu generieren.

\subsection{Zusammenfassung}
Am Ende des Rechenprozesses werden die wichtigsten Daten nochmal kurz zusammengefasst und der Mittelwert sowie die Randwerte unseres Ausgleichshistogramms aufglistet.



\end{document}
