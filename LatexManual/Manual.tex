\documentclass[14pt,a4paper]{scrartcl}
 
\usepackage[utf8]{inputenc}
\usepackage[T1]{fontenc}
\usepackage{lmodern}
\usepackage[ngerman]{babel}
\usepackage{amsmath}
 
\title{Manual}
\author{Frederik und Henrik}
\date{\today}
\begin{document}
 
\maketitle
\tableofcontents
\section{Inhalt}
ProjektarbeitDiffusion C++ Projekt:
\newline
\textbf{Sourcedateien}:
Teilchen.h;
Teilchen.cpp;
Kasten.h;
Kasten.cpp;
Teilchengernerator.h;
Teilchengernerator.cpp;
Ausgleich.h;
Ausgleich.cpp;
Eingaben.h;
Eingaben.cpp;
Grafikoutput.h;
Grafikoutput.cpp;
ProjektarbeitDiffusion.cpp;

\section{Diffusion}
Unser Programm soll den Diffusionsvorgang veranschaulichen.
Es werden wahlweise in einem Kasten, der durch eine Wand mit einem Spalt in zwei Hälften unterteilt ist, zufällig oder manuell Teilchen erzeugt.

\section{Programm starten}
Insallieren sie das Programm mit CMake und führen sie die Executable aus.
Folgen sie nun den Instruktionen des Programms.
\section{Features}
  



\end{document}
