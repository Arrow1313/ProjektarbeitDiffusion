\documentclass{beamer}

\usepackage[utf8]{inputenc}
\usepackage[T1]{fontenc}
\usepackage[ngerman]{babel}

\usepackage{dsfont}

\usepackage{graphicx}

\usepackage{amsmath}
\usepackage{amssymb}
\usepackage{amsfonts}

\usepackage{microtype}
\usepackage{lmodern}

\usetheme{Goettingen}  %% Themenwahl
 
\title{Diffusion}
\author{Frederik Strothmann und Henrik Jürgens}
\date{\today}
 
\begin{document}
\maketitle
\frame{\tableofcontents[currentsection]}
 


\section{Generelles}

\begin{frame} %%Eine Folie
  \frametitle{Problemstellung} %%Folientitel
 \begin{itemize}
 		\item Diffusion in einem Kasten mit zwei Kammern
        \item N Teilchen
        \item Wechselwirkung durch zentralen Stoß
\end{itemize}
\end{frame}

\begin{frame} %%Eine Folie
  \frametitle{Motivation} %%Folientitel
 \begin{itemize}
 		\item Untersuchen von Verteilungsprozessen in Abhängigkeit von:
 		\begin{itemize}
 			\item Kastengröße
 			\item Spaltgröße
 			\item Anzahl/Größe/Masse der Teilchen
 		\end{itemize}
\end{itemize}
\end{frame}


\begin{frame} %%Eine Folie
  \frametitle{Aktueller Programm-Umfang} %%Folientitel
 \begin{itemize}
 		\item Kasten beliebig einstellbar
 		\item Teilchen automatisch generieren oder von Hand erstellbar
 		\item Plotten der Teilchen bahnen
 		\item Plotten der Teilchenanzahl pro Kammer gegen die Zeit
 		\item Plotten der Verteilung der Iterationsdauern
 		\item Auswertung der Iterationsdauern (Min,Max,Mittelwert)
\end{itemize}
\end{frame}



\section{Aufbau des Programms}

\begin{frame} %%Eine Folie
  \frametitle{UML-Diagramm} %%Folientitel
  \begin{figure}[htb]
		\centering
		\includegraphics[scale = 0.17]{UML-Projekarbeit.png}
		\caption{UML-Diagramm des Programms}
  \end{figure}
\end{frame}

\begin{frame} %%Eine Folie
  \frametitle{Flussiagramm der main.cpp} %%Folientitel
  \begin{figure}[htb]
		\centering
		\includegraphics[scale = 0.16]{Flussdiagramm_main.png}
		\caption{Flussiagramm des Programms}
  \end{figure}
\end{frame}



\end{document}